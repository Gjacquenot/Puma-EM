\chapter{MoM solution of the integral equations for PEC}
%
\par
Let us concentrate on perfect electric conductor (PEC), a very important case in practice. In this case, the fields are zero inside of the inhomogeneity, magnetic surface currents are zero, and the integral equations are given by rewriting (\ref{eqn:EFIE-O}) and (\ref{eqn:MFIE-O}) as follows
\begin{equation}\label{eqn:EFIE-O_PEC}
\boxed{
\field{E}^\text{inc} =  - \frac{1}{j \omega \varepsilon_1} \operator{D}_1 \left( \current{J}_{S}\right) 
} \qquad EFIE-O
\end{equation}
\begin{equation}\label{eqn:MFIE-O_PEC}
\boxed{
\field{H}^\text{inc} = - \frac{1}{2} \vect{\hat{n}} \times \current{J}_{S} - \operator{K}_1 \left(\current{J}_{S}\right)
} \qquad MFIE-O
\end{equation}
with
\begin{align*}
\operator{D}_1\arg{\current{J}_S} &= \left(\nabla \nabla \cdot + k_1^2\right) \int_{S} G_1\left(\vect{r}, \vect{r}'\right) \current{J}_S\left(\vect{r}'\right) d\vect{r}' \\
\operator{K}_1\left(\current{J}_S\right) &= \int_S \nabla G_1\left(\vect{r}, \vect{r}'\right) \times \current{J}_S\left(\vect{r}'\right) d\vect{r}'.
\end{align*}


\section{Method of moments formulation}

\subsection{Basis functions: definition and properties}
\label{subsubsec:Basis functions: definition and properties}
%
\par
First the surface of the body is discretized by triangular patches. Then the electric current are approximated by a sum of RWG \emph{basis functions} such that, for any point on the surface of the body, we have
\begin{equation}\label{eqn:current_decomposition}
\vect{J}\arg{\vect{r}} \simeq \sum_{n = 1}^{N} I_n \vect{f}_n \arg{\vect{r}}
\end{equation}
where $\vect{f}_n \arg{\vect{r}}$ are triangular edge-defined basis functions called RWG functions, first introduced by Rao, Wilton and Glisson in \cite{Rao_82}. $N$ is the number of basis functions. The $n^\text{th}$ RWG basis function is defined on the adjacent triangles associated with edge $n$, and is given by \cite{Michalski_90}:
\begin{equation} \label{eqn:RWG definition}
\vect{f}_n \arg{\vect{r}} \triangleq
\begin{cases}
\vect{f}_n^+ = \frac{l_n}{2 A_n^+} \left( \vect{r} - \vect{r}_n^+\right), & \quad \text{$\vect{r}$ in $T_n^+$} \\
\vect{f}_n^- = -\frac{l_n}{2 A_n^-} \left( \vect{r} - \vect{r}_n^-\right), & \quad \text{$\vect{r}$ in $T_n^-$} \\
\vect{0}, & \quad \text{otherwise}
\end{cases}
\end{equation}
in which $l_n$ is the length of the edge, $A_n^\pm$ is the area of triangle $T_n^\pm$ and $\left( \vect{r} - \vect{r}_n^\pm\right)$ is the position vector in the triangle plane and relative to the node opposed to the edge.
%
\par
These RWG basis functions display properties that are very useful in the MoM, among which the most important are given hereafter \cite{Rao_82}:
\begin{enumerate}
\item the component of current normal to the $n^\text{th}$ edge is constant and continuous across the edge;
\item all edges of $T_n^+$ and $T_n^-$ are free of line charges;
\item the surface divergence of $\vect{f}_n$, proportional to the surface charge density associated with the basis element, is
\begin{equation} \label{eqn:RWG divergence}
\nabla_S \cdot \vect{f}_n \arg{\vect{r}} =
\begin{cases}
\frac{l_n}{A_n^+}, & \quad \text{$\vect{r}$ in $T_n^+$} \\
-\frac{l_n}{A_n^-}, & \quad \text{$\vect{r}$ in $T_n^-$} \\
0, & \quad \text{otherwise}.
\end{cases}
\end{equation}
The charge density is constant in each triangle and the total charge associated with the triangle pair $T_n^+$ and $T_n^-$ is zero;
\item a linear superposition of the three basis functions associated with a triangle can represent a linear current flowing through this triangle in an arbitrary direction.
\end{enumerate}

\subsection{Discretization and testing of the integral equations}
%
\par
With (\ref{eqn:current_decomposition}), (\ref{eqn:EFIE-O_PEC}) and (\ref{eqn:MFIE-O_PEC}) become:
\begin{equation}\label{eqn:EFIE-O_PEC_1}
\field{E}^\text{inc} =  - \frac{1}{j \omega \varepsilon_1} \sum_{n=1}^{N} I_n \left(\nabla \nabla \cdot + k_1^2\right) \int_{D_n} G_1\left(\vect{r}, \vect{r}'\right) \current{f}_n\left(\vect{r}'\right) d\vect{r}'
\end{equation}
\begin{equation}\label{eqn:MFIE-O_PEC_1}
\field{H}^\text{inc} = - \frac{1}{2} \sum_{n=1}^{N} I_n \vect{\hat{n}} \times \current{f}_{n}\arg{\vect{r}} - \sum_{n=1}^{N} I_n \int_{D_n} \nabla G_1\left(\vect{r}, \vect{r}'\right) \times \current{f}_n\left(\vect{r}'\right) d\vect{r}'
\end{equation}
where $D_n$ is the domain of basis function $\vect{f}_n$.
%
\par
Let us now test the integral equations as per \cite{Har_68}, \cite{Rao_82}. Testing is most often done with help of the RWG basis functions, in which case we are in a Galerkin numerical scheme.  However, testing with $\uvect{n} \times \text{RWG}$ functions must also be considered when using the CFIE. The testing function, denoted by $\vect{g}_m$, can therefore be:
\begin{equation}\label{eqn:testing_function}
\vect{g}_m \triangleq \vect{f}_m \qquad \text{or} \qquad \vect{g}_m \triangleq \uvect{n} \times \vect{f}_m \qquad \text{on} \; D_m
\end{equation}
where $\uvect{n}$ is the outward normal to the surface, and where $D_m$ is the domain of the test function $\vect{g}_m$, defined in the same manner as for $\vect{f}_m$.
%
\par
Testing (\ref{eqn:EFIE-O_PEC_1}) and (\ref{eqn:MFIE-O_PEC_1}) with $\vect{g}_m$ yields
\begin{equation}\label{eqn:EFIE-O_PEC_2}
\boxed{-\int_{D_m}\vect{g}_m \arg{\vect{r}} \cdot \field{E}^\text{inc} d\vect{r} =  \frac{1}{j \omega \varepsilon_1} \sum_{n=1}^{N} I_n \Bigg[\underbrace{\int_{D_m}\vect{g}_m \arg{\vect{r}} \cdot \left( \left(\nabla \nabla \cdot + k_1^2\right) \int_{D_n} G_1\left(\vect{r}, \vect{r}'\right) \current{f}_n\left(\vect{r}'\right) d\vect{r}' \right) d\vect{r}}_{D_{mn}^{(1)}} \Bigg]}
\end{equation}
\begin{equation}\label{eqn:MFIE-O_PEC_2}
\boxed{-\int_{D_m}\vect{g}_m \arg{\vect{r}} \cdot \field{H}^\text{inc} d\vect{r} = \sum_{n=1}^{N} I_n \Bigg[\underbrace{\int_{D_m} \frac{1}{2} \vect{g}_m \arg{\vect{r}} \cdot \vect{\hat{n}} \times \current{f}_{n}\arg{\vect{r}}d\vect{r} }_{J_{mn}}+ \underbrace{\int_{D_m} \vect{g}_m \arg{\vect{r}} \cdot \left(\int_{D_n} \nabla G_1\left(\vect{r}, \vect{r}'\right) \times \current{f}_n\left(\vect{r}'\right) d\vect{r}'\right) d\vect{r}}_{K_{mn}^{(1)}} \Bigg]}.
\end{equation}
We can recast the above equations in matrix form:
\begin{equation}
\frac{1}{j \omega \varepsilon_1} \underline{\underline{D}}^{(1)} \underline{I} = \underline{V}^{E}, \quad \left[\underline{\underline{J}} + \underline{\underline{K}}^{(1)} \right] \underline{I} = \underline{V}^{H}
\end{equation}
where
\begin{itemize}
\item $D_{mn}^{(1)} = \int_{D_m}\vect{g}_m \arg{\vect{r}} \cdot \left( \left(\nabla \nabla \cdot + k_1^2\right) \int_{D_n} G_1\left(\vect{r}, \vect{r}'\right) \current{f}_n\left(\vect{r}'\right) d\vect{r}' \right) d\vect{r}$
\item $K_{mn}^{(1)} = \int_{D_m} \vect{g}_m \arg{\vect{r}} \cdot \left(\int_{D_n} \nabla G_1\left(\vect{r}, \vect{r}'\right) \times \current{f}_n\left(\vect{r}'\right) d\vect{r}'\right) d\vect{r}$
\item $J_{mn} = \int_{D_m} \frac{1}{2} \vect{g}_m \arg{\vect{r}} \cdot \vect{\hat{n}} \times \current{f}_{n}\arg{\vect{r}}d\vect{r}$
\item $V_m^{E} = -\int_{D_m}\vect{g}_m \arg{\vect{r}} \cdot \field{E}^\text{inc} d\vect{r}$
\item $V_m^{H} = -\int_{D_m}\vect{g}_m \arg{\vect{r}} \cdot \field{H}^\text{inc} d\vect{r}$.
\end{itemize}
%
\par
The above elements will depend upon the testing function used, see (\ref{eqn:testing_function}). They will be denoted with a $t$ superscript if $\vect{g}_m \triangleq \vect{f}_m$ are used, with a $n$ superscript if $\vect{g}_m \triangleq \uvect{n} \times \vect{f}_m$ are used.
%
\par
Two last remarks are in order for matrix elements $K_{mn}$ and $J_{mn}$:
\begin{itemize}
\item $K_{mn} = 0$ if $\vect{g}_m$ and $\vect{f}_n$ are coplanar
\item $J_{mn} = 0$ if $\vect{g}_m = \vect{f}_m$, and $\vect{f}_n = \vect{f}_m$.
\end{itemize}

